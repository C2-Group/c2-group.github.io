\documentclass[tikz]{standalone}%standalone
%\documentclass[11pt]{ctexart}
%\documentclass〔12pt,dvipdfm]{article}


%\documentclass[a4paper,10pt]{article}

%\documentclass[a4paper,10pt]{ctexart} % instead of `article'

\usepackage{tikz}
\usetikzlibrary{arrows,chains,matrix,positioning,scopes,graphs}

% \usepackage{smartdiagram}
% \usepackage{forest}
% \usepackage{tikz-qtree}

\usepackage{xeCJK}%根据自己的需要加载宏包


%\setmainfont{Caladea}

\makeatletter

% \tikzset{join/.code=\tikzset{after node path={%
% \ifx\tikzchainprevious\pgfutil@empty\else(\tikzchainprevious)%
% edge[every join]#1(\tikzchaincurrent)\fi}}}

% \tikzset{every tree node/.style={align=center,anchor=north}}


\makeatother
%
% \tikzset{>=stealth',every on chain/.append style={join},
%          every join/.style={->}}
% \tikzstyle{labeled}=[execute at begin node=$\scriptstyle,
%    execute at end node=$]
%
\begin{document}

\begin{tikzpicture}
    \graph[nodes={draw, fill=green},edges={thick,red},no placement] { %grow right sep=1.5cm,branch down=1.5cm,
     开始[fill=yellow, x=0,y=0] -> {
        任务1[x=3,y=2] -> 任务7[x=6,y=1.5] -> {任务10[x=9,y=1.5],任务11[x=9,y=0.5]},
        任务2[x=3,y=1] -> 任务8[x=6,y=0] -> 任务11,
        任务3[x=3,y=0] -> 任务7 -> 任务10,
        任务4[x=3,y=-1] -> 任务8 -> 任务13[x=9,y=-1.5],
        任务5[x=3,y=-2] -> 任务9[x=6,y=-1.5] -> {任务12[x=9,y=-0.5],任务13}
     } -> 结束[fill=yellow, x=12,y=0];
    };
\end{tikzpicture}


% \begin{tikzpicture}
% \graph [grow down,branch right=2.5cm]
% { root -> {
%     child 1,
%     child 2 -> {
%       grand child 1,
%       任务4},
%     child 3 -> {
%       grand child 3,
%       任务4} } -> 结束
% };
% \end{tikzpicture}

\end{document}
